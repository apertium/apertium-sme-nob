% Created 2009-09-15 Tue 22:47
\documentclass[11pt]{article}
\usepackage{freerbmt09}
\usepackage[utf8x]{inputenc}
\usepackage{times}
\usepackage{natbib}
\usepackage{url}
\usepackage[small,bf]{caption}
\usepackage{linguex}
\usepackage{latexsym}
%\usepackage{setspace}

\usepackage{hyperref}

\title{apertium-sme-nob: using an HFST lexicon in a long-distance RBMT system} % ugh TODO

%\author{NN\\  X \\ X \\  X \\  {\tt \small   X@X} \And  NN \\  X \\  X \\  X \\    {\tt \small  X@X}}
\author{N.N.\\  Department of X \\ University of Y\\  Y, Z \\  {\tt \small   email@example.com} % \And  Trond Trosterud \\  Department of Linguistics \\  University of Tromsø \\  Tromsø, Norway \\    {\tt \small  trond.trosterud@uit.no}
}


\newcommand{\comment}[1]{\textbf{SKRIV} {\it #1}}
%\renewcommand{\comment}[1]{} % uncomment for final version

\begin{document}

\maketitle

  \begin{abstract}
    We describe the development of a rule-based machine translation
    system from Northern Sámi to Norwegian Bokmål built on a
    combination of Free and Open Source resources: the Apertium
    platform and the Giellatekno HFST lexicon and Constraint Grammar
    disambiguator.
    \comment{more}
    We detail the integration of these and other resources in the
    system along with the construction of the lexical and structural
    transfer, and evaluate the translation quality in comparison with
    another system. Finally, some future work is suggested.
  \end{abstract}

\section{Introduction}
Northern Sámi (sme) is an Uralic language spoken by about 20,000
people in the northern parts of Norway, Sweden and Finland. Norwegian
Bokmål (nob) is a North Germanic language with about 4.5 speakers,
mostly in Norway.

Most sme speakers in Norway understand nob, while most nob speakers do
not understand sme. The languages are completely unrelated, and the
linguistic distance is great, making it hard to achieve high quality.
A nob→sme system would only be useful if the quality were good enough
that it could be used for text production (post-editing). On the other
hand, a sme→nob gisting-quality system can be useful for the large
group of nob speakers who don't understand sme. Thus we chose to focus
on the sme→nob direction first.




\section{Design}
 \label{sec:design}
\subsection{The Apertium Pipeline}
\subsection{HFST}


\section{Development}

  \label{sec:development}
\subsection{Resources}

\subsection{Analysis and generation}

\subsection{Disambiguation}
\subsection{Lexical selection}

\subsection{Lexical transfer}

\subsection{Structural transfer}
\label{sec:structural-transfer}

\section{Evaluation}
\label{sec:eval}



\subsection{Word Error Rate on Post-Edited text}
\label{sec:WER}
\comment{or at least on scraped web sites ...}
\subsection{Gisting eval}


\subsection{Error analysis}

\section{Discussion and outlook}

\section*{Acknowledgements}
Development was funded by \comment{what are they called now?}
Thanks to N.N.


\nocite{zubizarreta2009amt}

\bibliographystyle{apalike}
\bibliography{apertium}

\end{document}
