\newcommand{\comment}[1]{\textbf{SKRIV} {\it #1}}
%\renewcommand{\comment}[1]{} % uncomment for final version

\newcommand{\href}[2]{{\tt #1}} % put hyperref in preamble? TODO

\achapter{apertium-sme-nob:using an HFST lexicon in a long-distance RBMT system}{N.N.}


  
    % \begin{abstract}
    We describe the development of a rule-based machine translation
    system from Northern Sámi to Norwegian Bokmål built on a
    combination of Free and Open Source (FOSS) resources: the Apertium
    platform and the Giellatekno HFST lexicon and Constraint Grammar
    disambiguator.
    \comment{more}
    We detail the integration of these and other resources in the
    system along with the construction of the lexical and structural
    transfer, and evaluate the translation quality in comparison with
    another system. Finally, some future work is suggested.
    % \end{abstract}




\section{Introduction}
Northern Sámi (sme) is an Uralic language spoken by about 20,000
people in the northern parts of Norway, Sweden and Finland. Norwegian
Bokmål (nob) is a North Germanic language with about 4.5 speakers,
mostly in Norway.

Most sme speakers in Norway understand nob, while most nob speakers do
not understand sme. The languages are completely unrelated, and the
linguistic distance is great, making it hard to achieve high quality.
A nob$\rightarrow{}$sme system would only be useful if the quality
were good enough that it could be used for text production
(post-editing). On the other hand, a sme$\rightarrow{}$nob
gisting-quality system can be useful for the large group of nob
speakers who don't understand sme. Thus we chose to focus on the
sme$\rightarrow{}$nob direction first.


\section{Design}
 \label{sec:design}



\subsection{The Apertium Pipeline}
Apertium is a highly modular, shallow-transfer pipeline MT system. The
typical pipeline consist of source language analysis with a finite
state transducer (FST), followed by disambiguation using a Hidden
Markov Model (HMM) and/or Constraint Grammar (CG), then lexical
transfer (word-translation on the disambiguated source) and one or
more levels of structural transfer (ie.~reordering and changes to
morphological features), followed by target-language generation with
an FST.

Most Apertium language pairs use the lttoolbox FST package for
analysis and generation; lttoolbox dictionaries are written in XML,
where one dictionary be compiled both to an analyser and a generator.
This pair uses lttoolbox for nob generation, but the sme analyser is
written in the Xerox lexc/twol format\citep{beesley2003fsm}. We use
the FOSS package Helsinki Finite State Tools (HFST) \comment{cite} to
compile and run the analyser (see section \ref{sec:hfst}).

The morphological analysis gives us ambiguous output with no syntactic
information. For morphological (e.g.~part-of-speech) disambiguation,
syntactic annotation/disambiguation and lexical
selection\footnote{Similar to Word Sense Disambiguation, but
  restricted to senses that have differing translations.}, we use
Constraint Grammar \citep{karlsson1990cgf}\footnote{Using the FOSS
  package {\tt \small VISL CG-3},
  \href{http://beta.visl.sdu.dk/cg3.html}{http://beta.visl.sdu.dk/cg3.html}}. Morphological
disambiguation and syntactic annotation/disambiguation are run as one
CG module, the output of which is unambiguous both morphologically
(one analysis per form) and syntactically (each form/analysis is
annotated with one syntactic tag). 

The first CG module is directly followed by a lexical selection CG
module\footnote{If the disambiguation rules leaves any ambiguity, that
  module is configured only print the first analysis. We may later
  train an HMM to get rid of leftover ambiguity, this would go between
  the two CG modules.}, which may add subscripts to lemmas in certain
contexts in order to select a different lexical translation. 

Lexical selection is followed by pretransfer (minor format changes in
preparation of transfer) and then a four-stage finite-state-based
chunking transfer. The first stage handles lexical transfer as well as
chunking based on patterns of morphological and syntactic tags
(further detail in section \ref{sec:structural-transfer}).

Output from the transfer module is fed to morphological generation
with the lttoolbox-based nob generator. De-/reformatters applied to
the beginning and end of the pipeline let us preserve formatting of
various document types.

\subsection{HFST}
\label{sec:hfst}
One novel feature of apertium-sme-nob is the HFST-based analyser. HFST
makes it possible to compile lexicons and morphologies originally
written for the closed-source Xerox Finite State Tools using only free
and open source tools, and run them with Apertium-compatible output
formats. The sme analyser is written in the Xerox-based lexc and twol
formalisms. HFST analysers are slower at compiling and processing than
lttoolbox, but certain morphological phenomena are impossible or at
least very hard to describe in lttoolbox. Since Northern Sámi is quite
morphologically complex, a purely lttoolbox-based analyser would be
difficult to maintain.

\section{Development}
  \label{sec:development}

This section describes how the language pair was developed.
\subsection{Resources}
We re-used several FOSS resources in creating this language pair. The
nob generator came from {\tt apertium-nn-nb}\citep{unhammer2009rfr},
while most of the sme resources came from N.N.\footnote{See
  \href{http://example.com}{http://example.com}.} \comment{Sámi
  Giellatekno at the University of Tromsø}, including the
lexicon/morphology and disambiguator/syntactic annotation CG. These
were continually updated as the ``upstream'' versions changed.

\subsection{Analysis}
As mentioned, the 

\subsection{Disambiguation}
\subsection{Lexical selection}
The lexical selection CG was written from scratch.

\subsection{Lexical transfer}

\subsection{Structural transfer}
\label{sec:structural-transfer}

\subsection{Generation}

\section{Evaluation}
\label{sec:eval}



\subsection{Word Error Rate on Post-Edited text}
\label{sec:WER}
\comment{or at least on scraped web sites ...}
\subsection{Gisting eval}
\comment{The idea here is that if the user can select the right
  alternative, she must have some comprehension of the meaning. Now,
  the sentence might have errors such that both the 'right'
  alternative and the context are wrong in the same way, but if errors
  are random enough, and the words that are translated within a
  sentence are never consistently skewed towards the same wrong
  meaning, selecting the right alternative will be a sign of
  comprehending the meaning.}

\subsection{Error analysis}

\section{Discussion and outlook}

\section*{Acknowledgements}
Development was funded by \comment{what are they called now?}
Thanks to N.N.


\nocite{zubizarreta2009amt}
