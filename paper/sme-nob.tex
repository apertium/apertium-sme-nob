\newcommand{\comment}[1]{\textbf{SKRIV} {\it #1}}
%\renewcommand{\comment}[1]{} % uncomment for final version


\achapter{apertium-sme-nob:using an HFST lexicon in a long-distance RBMT system}{The Author}


  
    % \begin{abstract}
    We describe the development of a rule-based machine translation
    system from Northern Sámi to Norwegian Bokmål built on a
    combination of Free and Open Source (FOSS) resources: the Apertium
    platform and the Giellatekno HFST lexicon and Constraint Grammar
    disambiguator.
    \comment{more}
    We detail the integration of these and other resources in the
    system along with the construction of the lexical and structural
    transfer, and evaluate the translation quality in comparison with
    another system. Finally, some future work is suggested.
    % \end{abstract}




\section{Introduction}
Northern Sámi (sme) is an Uralic language spoken by about 20,000
people in the northern parts of Norway, Sweden and Finland. Norwegian
Bokmål (nob) is a North Germanic language with about 4.5 speakers,
mostly in Norway.

Most sme speakers in Norway understand nob, while most nob speakers do
not understand sme. The languages are completely unrelated, and the
linguistic distance is great, making it hard to achieve high quality.
A nob$\rightarrow{}$sme system would only be useful if the quality
were good enough that it could be used for text production
(post-editing). On the other hand, a sme$\rightarrow{}$nob
gisting-quality system can be useful for the large group of nob
speakers who don't understand sme. Thus we chose to focus on the
sme$\rightarrow{}$nob direction first.


\section{Design}
 \label{sec:design}



\subsection{The Apertium Pipeline}
Apertium is a highly modular, shallow-transfer pipeline MT system. The
typical pipeline consist of source language analysis with a finite
state transducer (FST), followed by disambiguation using a Hidden
Markov Model (HMM) and/or Constraint Grammar (CG), then lexical
transfer (word-translation on the disambiguated source) and one or
more levels of structural transfer (ie.~reordering and changes to
morphological features), followed by target-language generation with
an FST.

Most Apertium language pairs use the lttoolbox FST package for
analysis and generation; lttoolbox dictionaries are written in XML,
where one dictionary be compiled both to an analyser and a generator.
This pair uses lttoolbox for nob generation, but the sme analyser is
written in the Xerox lexc/twol format\citep{beesley2003fsm}. We use
the FOSS package HFST\comment{cite} to compile and run the analyser
(see sec.~\ref{sec:hfst}).

For disambiguation, syntactic annotation and lexical selection, we use
Constraint Grammar \citep{karlsson1990cgf}\footnote{Using {\tt \small
    VISL CG-3}, \tt{http://beta.visl.sdu.dk/cg3.html} \comment{href
    todo}}

This is followed by
pretransfer, four-stage chunk-based transfer and lttoolbox generation.



Dictionaries written in either the lttoolbox XML format or the Xerox
lexc format are compiled into finite state transducers, so that
word-for-word translations are in principle possible in both
directions using only two monolingual dictionaries (morphological
analysis/generation) and one translational (transfer) dictionary.



For disambiguation and syntactic annotation, we use Constraint
Grammar, CG. A CG
\citep{karlsson1990cgf} consists of hand-written rules \comment{...}

If the CG leaves us ambiguity, we simply choose the first
analysis\footnote{We plan on training an HMM to get rid of leftover
  ambiguity}.


The transfer module is finite state based and handles three-stage
chunking transfer, although we so far only use one-stage transfer,
applying operations directly on patterns of morphological categories
(described in further detail in section
\ref{sec:structural-transfer}). Output from the transfer module is fed
to morphological generation. De-/reformatters applied to the beginning
and end of the pipeline let us preserve formatting of various document
types.

\subsection{HFST}
\label{sec:hfst}

\section{Development}

  \label{sec:development}
\subsection{Resources}

\subsection{Analysis and generation}

\subsection{Disambiguation}
\subsection{Lexical selection}

\subsection{Lexical transfer}

\subsection{Structural transfer}
\label{sec:structural-transfer}

\section{Evaluation}
\label{sec:eval}



\subsection{Word Error Rate on Post-Edited text}
\label{sec:WER}
\comment{or at least on scraped web sites ...}
\subsection{Gisting eval}
\comment{The idea here is that if the user can select the right
  alternative, she must have some comprehension of the meaning. Now,
  the sentence might have errors such that both the 'right'
  alternative and the context are wrong in the same way, but if errors
  are random enough, and the words that are translated within a
  sentence are never consistently skewed towards the same wrong
  meaning, selecting the right alternative will be a sign of
  comprehending the meaning.}

\subsection{Error analysis}

\section{Discussion and outlook}

\section*{Acknowledgements}
Development was funded by \comment{what are they called now?}
Thanks to N.N.


\nocite{zubizarreta2009amt}
